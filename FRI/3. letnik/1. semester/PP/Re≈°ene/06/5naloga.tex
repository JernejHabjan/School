\documentclass[11pt,a4paper]{article}

\usepackage[utf8x]{inputenc}   % omogoča uporabo slovenskih črk kodiranih v formatu UTF-8
\usepackage[slovene]{babel}    % naloži, med drugim, slovenske delilne vzorce

\usepackage[hyphens]{url}
\usepackage{hyperref}



\title{Uporaba in analiza Monte-Carlo drevesnega preiskovanja na strateški igri}
\author{Jernej Habjan\\
jh0228@student.uni-lj.si\\
\ \\
predvideni MENTOR: doc. dr. Matej Guid \\
Fakulteta za računalništvo in informatiko Univerze v Ljubljani
\date{\today}         
}



\begin{document}
\maketitle

\section{Najbolj relevantna publikacija mojega predvidenega mentorja v zvezi z mojo predvideno diplomsko nalogo}

Realno-strateške igre so pri algoritmiki predvsem znane po zelo velikem razvejanju, saj so igre veliko daljše in prisotnih je več figur.
Tako da je abstrakcija pri takih igrah obvezna, saj ne moremo preiskati celega preiskovalnega drevesa, vendar le del.
Abstrakcija lahko poteka na način, da združimo posamezne enote v skupine, in potem ukazujemo skupinam in ne posameznim enotam.
Prav tako lahko posamezne akcije nadgradimo v več nivojske.
Primer za to bi bil gradnja hiše.
Pri gradnji hiše mora oseba priti na določeno lokacijo, vzeti surovine, graditi hišo in vmes opravljati še druge ukaze.
To zaporedje ukazov lahko posplošimo na večnivojskega in obravnavamo skupek ukazov kot enega samega.\\
V konferenčnem papirju, pri katerem je sodeloval mentor doc. Matej Guid je pa opisan drugačen način izboljšanja iskanja rešitve v preiskovalnem drevesu~\cite{stoiljkovikj2015computational}.
K problemu pristopi na način človeškega razmišljanja, saj eksperti šaha (papir je predvsem osredotočen na šah)  na podlagi predznanja in izkušenj izgradijo t.i. \textit{smiselna drevesa}.\\
Smiselna drevesa so veliko manjša od splošnih preiskovalnih dreves, saj omejimo akcije, ki jih izvedemo na majhno podmožico.
Akcija je smiselna, če je njena ocen večja kot pričakovana, ali pa če je napaka pri izvedbi akcije manjša kot nasprotnikova pričakovana napaka.


\section{Katere so tri najbolj citirane publikacije mojega predvidenega mentorja}

\subsection{V sistemu COBISS oziroma SICRIS}
V COBISS oziroma SICRIS so najbolj citirana naslednja dela:
\begin{description}
\item Computer analysis of world chess champions, kjer ima publikacija WoS 12,
\item How trustworthy is CRAFTY'S analysis of world chess champions?, kjer ima publikacija WOS 7,
\item Using heuristic-search based engines for estimating human skill at chess, kjer ima publikacija WOS 6.

\end{description}

\subsection{V Google učenjaku}
Najbolj citiranje mentorjeve publikacije v Google učenjaku:
\begin{description}
	\item Computer analysis of world chess champions ima 37 citatov,
	\item Fighting Knowledge Acquisition Bottleneck with Argument Based Machine Learning ima 25 citatov,
	\item How trustworthy is Crafty’s analysis of world chess champions ima 22 citatov.
\end{description}

\section{Kakšen je h-indeks mojega predvidenega mentorja}

\subsection{V sistemu SICRIS}
V Google učenjaku ima doc. Matej Guid h-index 5.
\subsection{V Google učenjaku}
V Google učenjaku ima doc. Matej Guid h-index 10.

\section{Strani mentorja na Google učenjaku, na Research gate in v  Academiji}
Docenta Mateja Guida najdemo na naslednjih povezavah:
\begin{description}
\item[Google učenjak:] \url{https://goo.gl/grjYCF}
\item[Research gate:] \url{https://www.researchgate.net/profile/Matej_Guid}
\item[Academia:] \url{http://independent.academia.edu/MatejGuid}

\end{description}

\bibliographystyle{plain}
\bibliography{literatura}

\end{document}  




