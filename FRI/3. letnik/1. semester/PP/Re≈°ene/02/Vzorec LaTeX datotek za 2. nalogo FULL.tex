\documentclass[11pt,a4paper]{article}

\usepackage[utf8x]{inputenc}   % omogoča uporabo slovenskih črk kodiranih v formatu UTF-8
\usepackage[slovene]{babel}    % naloži, med drugim, slovenske delilne vzorce

\usepackage{graphicx} % omogoča vključevanje slik


\title{Upraba Monte-Carlo drevesnega preiskovanja v strateških igrah}
\author{Jernej Habjan\\
jh0228@student.uni-lj.si\\
\ \\
možni MENTOR: doc. Matej Guid \\
Fakulteta za računalništvo in informatiko Univerze v Ljubljani
\date{\today}         
}


\begin{document}
\maketitle

\section{Motiv za diplomsko nalogo}
Igranje realno-časovnih strateških iger je za računalnik velik problem, saj je možnosti v vsakem trenutku tako veliko, da preprosto ne more računalnik preiskati celega prostora.
Kombinacij je veliko več kot pri igranju šaha ali igre Go, saj je veliko več možnosti, kaj posamezen osebek lahko v igri naredi.

\section{Cilj diplomske naloge}
Cilj diplomske naloge je implementacija in vrednotenje algoritma Monte-Carlo drevesno preiskovanje na igri Trump Defense 2020, narejeni v pogonu Unreal Engine 4.\\


Računalnika, ki igra proti človeškem nasprotniku lahko sprogramiramo na klasičen način pogoj - akcija.
Tako so narejene klasične realno-časovno strateške igre, vendar gre veliko časa za implementacijo posameznih sovražnikovih napadov in nabiranju virov.\\
Lahko pa razvijemo algoritem, ki sam ugotovi, katera je optimalna odločitev v določeni situaciji in jo za tem izvede.\\
Na preprostih problemih z malo odločitvami in kratkimi igrami kot je igra križci in krožci, lahko izberemo preprostejše algoritme kot je algoritem Minimax, kjer igralec hoče povečati svojo možnost zmage, nasprotnik pa mu hoče to možnost zmanjšati.\\
Obstajajo tudi algoritmi, ki se naučijo igre iz učne množice, ki predstavlja nekaj iger, pri tem pa računalnik ugotovi zakonitosti igre in način igranja. Taki algoritmi so nevronske mreže ali globoke nevronske mreže, ki pa so zelo računsko potratni, vendar vračajo izjemne rezultate. Nevronske mreže je uporabila Andreja Kovačič za prepoznavanje prometnih znakov \cite{diploma1}.

Lahko pa uporabimo algoritem Monte-Carlo drevesno preiskovanje, ki deluje na principu hevrističnega preiskovanja prostora, kjer imamo prostor stanj (graf, drevo), množico dosegljivih stanj in povezave med stanji - glej sliko \ref{fig:test}.\\
Algoritem bom uporabil na svoji igri narejeni v celostnem pogonu Unreal Engine 4 imenovani Trump Defense 2020.

\section{Celostni pogon Unreal Engine 4}
Unreal Engine 4 je odprtokodni program podjetja Epic Games, ki je namenjen hitri izdelavi računalniških iger. Obstajajo še drugi celostni pogoni kot je Unity.\\
Razliko med tema pogonoma je dobro predstavil Marko Kladnik \cite{diploma2}.\\
Unreal Engine 4 omogoča hitro ustvarjanje iger s pomočjo posebnih diagramov (angl. blueprint) in hkrati podpira programski jezik C++, ki ga uporabimo za hitro izvedbo velikega števila matematičnih izrazov.\\
Pogon podpira odločitvena drevesa, v katera lahko sprogramiramo računalnikovo delovanje ob določenih pogojih, vendar je v igri Trump Defense 2020 narejena inteligenca računalnika s hevrističnim algoritmom Monte-Carlo drevesno preiskovanje.

\section{Monte-Carlo drevesno preiskovanje}
Algoritem je uporabljen za preiskovanje prostora možnosti, ki ga lahko posamezni osebki v igri izberejo.\\
Algitem deluje v štirih korakih vidnih na spodnji sliki:
\begin{figure}[htb]
	\begin{center}
		\includegraphics[width=0.8\columnwidth]{MonteCarloTreeSearch}
	\end{center}
	\caption{Na sliki so prikazani koraki algoritma izbira, širitev, simulacija in vračanje}
	\label{fig:test}
\end{figure}

Tako kot je Nejc Ilenič uporabil Monte-Carlo drevesno preiskovanje pri reševanju igre Scotland Yard, je algoritem uporabljen na igri Trump Defense \cite{diploma3}.

\cite{genetski}

\section{Preizkus rešitve}

Rešitev bom vrednotil v primerjavi z drugimi hevrističnimi algoritmi in spošno delovanje igre in moč računalnikovega igranje proti človeškim igralcem.

\begin{thebibliography}{9}
	
\bibitem{diploma1}
Andreja Kovačič.
Detekcija prometnih znakov s konvolucijskimi nevronskimi mrežami.
Diplomska naloga, Fakulteta za računalništvo in informatiko, Univerza v Ljubljani, 2017.

\bibitem{diploma2}
Marko Kladnik.
Primerjava igralnih pogonov Unity in Unreal Engine.
Diplomska naloga, Fakulteta za računalništvo in informatiko, Univerza v Ljubljani, 2015.

\bibitem{diploma3}
Nejc Ilenič.
Drevesno preiskovanje Monte Carlo pri namizni igri Scotland Yard. 
Diplomska naloga, Fakulteta za računalništvo in informatiko, Univerza v Ljubljani, 2015.

\bibitem{diploma4}
Rok Oblak.
Simulacija množic heterogenih agentov v realnem času.
Diplomska naloga, Fakulteta za računalništvo in informatiko, Univerza v Ljubljani, 2014.
\end{thebibliography}

\end{document}  




