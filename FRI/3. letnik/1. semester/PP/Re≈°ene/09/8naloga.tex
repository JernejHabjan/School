\documentclass[11pt,a4paper]{article}

\usepackage[utf8x]{inputenc}   % omogoča uporabo slovenskih črk kodiranih v formatu UTF-8
\usepackage[slovene]{babel}    % naloži, med drugim, slovenske delilne vzorce

\usepackage[hyphens]{url}
\usepackage{hyperref}

\usepackage{graphicx}

\title{Uporaba in analiza Monte-Carlo drevesnega preiskovanja na strateški igri}
\author{Jernej Habjan\\
jh0228@student.uni-lj.si\\
\ \\
predvideni MENTOR: doc. dr. Matej Guid \\
Fakulteta za računalništvo in informatiko Univerze v Ljubljani
\date{\today}         
}

\begin{document}
\maketitle

\section{Opis raziskovalne metode, ki je primerna za predlagano diplomsko temo}


Pri uporabi algoritma v realno-časovni igri bom uporabil študijo izvedljivosti.
Algoritem Monte-Carlo drevesno preiskovanje je že dokončan algoritem, ki se uporablja pri starteških igrah kot naprimer šah.
Tak algoritem, lahko uporabim pri realno-časovni strateški igri, vendar ga moram razširiti na več akcij vsako potezo.
Implementacija algoritma v igro bo študija izvedljivosti, ki bo pokazala rezultate hevrističnega algoritma za nadomestek tradicionalno-programirane umetne inteligence in s tem dokazala možno implementacijo tega algoritma v preproste nove strateške igre.

\section{Dela s podobno raziskovalno metodo}


Gal Kos je za diplomsko delo predstavil dogajanje v bojišču v okolju Unity~\cite{bojisce}. Sprva je moral dobiti informacije o enotah in standardih in jih aplicirati v okolje Unity. Soočen je bil tudi s prikazom 3D višinskih slik za generiranje 3D terena.
Njegovo delo je razdeljeno na pridobivanje standardov, kart in ostalih informacij, implementacijo aplikacije za 3D vizualizacijo na bojišču, kjer je te enote in sisteme prikazal in s tem dokazal prikaz kompleksnega sistema v okolju Unity.

Rok Oblak pa je bil postavljen z izzivom iskanja ustrezne stopnje kompleksnosti simuliranja agentov. Prav tako je tudi on to apliciral na celostni pogon, in sicer v Unity.
Za diplomsko delo je predstavil množico heterogenih agentov, kot zaznavanje okolja, izogibanje oviram itd.
Njegovo delo pa je razdeljeno na predstavitev okolja Unity, predstavitev navigacije in model agenta, implementacije agentov, končnih avtomatov v pogon in predstavitev rezultatov~\cite{heterogeni}.


Žiga Pirih pa je za diplomsko delo predstavil skladbe v 3D prostoru in je sprva pregledal teoretično podlago in obstoječe rešitve, potem pa izvedel vizualizacijo s tremi glavnimi deli: računanjem podrobnosti med skladbami, izdelavo 3D prostora ter prenašanje, pretakanje in predvajanje skladb v programu. S tem je predstavil skladbe kot sprehod po virtualni pokrajini in odkrivanje novih skladb na podlagi njihovih medsebojnih podobnostih. Prav tako je tudi on to predstavil v orodju Unity, za prenos skladb pa je uporabil programski jezik Python. Izvedel je študijo izvedljivosti in s tem dokazal da lahko s tako vizualizacijo ugotovimo podobnosti med skladbami~\cite{viz}.
\section{Seznam literature}

\bibliographystyle{plain}
\bibliography{literatura}

\end{document}  




