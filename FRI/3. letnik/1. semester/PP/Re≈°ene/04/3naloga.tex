\documentclass[11pt,a4paper]{article}

\usepackage[utf8x]{inputenc}   % omogoča uporabo slovenskih črk kodiranih v formatu UTF-8
\usepackage[slovene]{babel}    % naloži, med drugim, slovenske delilne vzorce
\usepackage{url}
\usepackage{breakurl}
\usepackage[breaklinks]{}
\def\UrlBreaks{\do\/\do-}


\title{Primeri plagiarizma v akademskem svetu}
\author{Jernej Habjan\\
jh0228@student.uni-lj.si\\
\
\\
Fakulteta za računalništvo in informatiko Univerze v Ljubljani
\date{\today}         
}


\begin{document}
\maketitle

\section{Plagiarizem pri slovenskih politikih}


Med slovenskimi politiki je bil obtožen plagiarizma magistrske naloge Borut Ambrožič, poslanec Pozitivne Slovenije. Magistrsko nalogo je prepisal od študentske ekonomske fakultete Klementine Jezeršek
~\cite{politiki_plagiarized}.
Očitke plagiarizma je vseskozi zavračal, saj naj bi bila raziskovalna tema slabo raziskana.

Posledica njegovega prepisovanja je bila izključitev iz stranke in pozneje odvzet naziv magistra znanosti.

\section{Ministrica Klavdija Markež}
Prav tako je plagiat magistrska naloga ministrice Klavdije Markež, ki vsebuje del diplomskega dela Jerneja Ladnika~\cite{markezeva_plagiarized}.
Test z antiplagiatorskim programom Eforus je zaprosila ministrica sama, ta pa je ugotovil 37-odstotno ujemanje z nenavedeno diplomsko nalogo.
Njena naloga, ki je po besedah poslank SDS Jelke Godec in Anje Bah Žibert skrivnost, pa v Markeževo presenečenost ni bila objavljena v sistemu COBIS, saj je bila prepričana, da jo je objavila fakulteta.

Ob ugotovitvi plagiarizma, ministrici grozi odstop, ki ga je pa sama ponudila Miru Cerarju, ta pa ga je sprejel~\cite{markezeva_finished_plagiarized}.
Prav tako pričakujejo odstop v Študentski organizaciji Slovenije.
Iz parlamenta je pa ne more prisiliti nihče, saj so ji volilci podelili poslanski mandat.

\section{Šerif David Clarke}
Šerif Milwaukeeja v Wisconsinu David Clarke, ki je magistriral v pomorski šoli v Montereyu, Kalifornija, v svojem delu ni pravilno navedel 47 del.
Dela je vključil kot literaturo, vendar jih ni pravilno citiral, saj jih ni označil z navednicami, in s tem označil, da avtorja citira.
Prav tako tudi on zanika plagiarizem, tako rekoč: "Samo nekdo, ki ima politični motiv bi rekel, da je to plagiat."~\cite{sherrif_plagiarized}.

Poročnik Clint Phillips je dejal: "Univerzitetni kodeks pri preverjanju plagiatorstva iz leta v leto spreminja.
V tem primeru ne bi mogli ugotoviti nobene kršitve, dolker ne bi izvršili popolne preiskave."
Tako da šerifu kazen ne sledi.


\bibliographystyle{plain}
\bibliography{literatura}

\end{document}  




