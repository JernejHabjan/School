\documentclass[11pt,a4paper]{article}

\usepackage[utf8x]{inputenc}   % omogoča uporabo slovenskih črk kodiranih v formatu UTF-8
\usepackage[slovene]{babel}    % naloži, med drugim, slovenske delilne vzorce

\usepackage{graphicx} % omogoča vključevanje slik


\title{Naslov možne diplomske naloge}
\author{Ime Priimek\\
e-naslov\\
\ \\
možni MENTOR: (doc./prof.) dr. X Y \\
Fakulteta za računalništvo in informatiko Univerze v Ljubljani
\date{\today}         
}


\begin{document}
\maketitle

Napišite dokument na najmanj eni strani in pol in na največ dveh straneh o možni diplomski temi na osnovi tega vzorca.
V seznamu literatura navedite vsaj štiri sorodne diplome FRI, ki jih najdete na spletišču eprints.fri.uni-lj.si.

Cilj te naloge je ugotoviti, kakšne so diplomske naloge na FRI, kakšne teme so možne in  katere teme pokrivajo posamezni mentorji.


\section{Motiv za diplomsko nalogo}

Zakaj je tematika pomembna, kaj je trenutno stanje, kako in komu bodo rezultati koristili?

Podobno tematiko obravnava diplomska naloga \cite{diploma1}.


\section{Kaj je konkretni cilj diplomske naloge}

Nova programska oprema, nova spletna storitev, za enega ciljnega uporabnika, za širšo javnost?


\section{S kakšnimi orodji bomo prišli do uresničitve cilja}

Kakšno opremo in orodja bomo uporabljali pri izdelavi diplomske naloge?


\section{Kako bomo preizkusili rešitev ali ustreza zadanim ciljem}

Demonstracija delovanja, testiranje s pomočjo bodočih uporabnikov, testiranje s pomočjo standardne podatkovne baze, formalni dokaz?


\section{Vključevanje slik}

Za idejo o svoji diplomski nalogi poiščite primerno sliko in jo vključite v dokument.
Na sliko se morate v besedilu tudi sklicevati, na primer takole, glej sliko \ref{fig:test}.


\begin{figure}[htb]
\begin{center}
\includegraphics[width=0.8\columnwidth]{LRV}
\end{center}
\caption{Primer vključitve slike v dokument}
\label{fig:test}
\end{figure}


\begin{thebibliography}{9}

\bibitem{diploma1}
Nataša Vodopivec.
Prenos meritev pametne zapestnice v enovito podatkovno bazo.
Diplomska naloga, Fakulteta za računalništvo in informatiko, Univerza v Ljubljani, 2016.

\end{thebibliography}

\end{document}  




