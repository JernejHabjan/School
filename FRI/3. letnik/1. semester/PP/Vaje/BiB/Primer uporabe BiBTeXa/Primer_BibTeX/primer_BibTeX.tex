\documentclass[12pt,a4paper]{article}
\usepackage[slovene]{babel}
\usepackage[hyphens]{url}
\usepackage{hyperref}


\title{Primer uporabe \textsc{Bib}\TeX}
\author{Franc Solina}

\date{\today}


\begin{document}
\newcommand{\sn}[1]{"`#1"'}

\maketitle


Z umetni\v skimi projekti se v Laboratoriju za ra\v cunalni\v ski vid ukvarjamo \v ze vrsto let \cite{leonardo,IS2001}, saj so tak\v sni projekti primerno testno okolje za metode ra\v cunalni\v skega vida \cite{poglavje_springer}.
Kot za vsako programsko opremo, ki se uporablja vrsto let, pa je potrebno take projekte tudi vzdr\v zevati \cite{ZKM}.

Na\v s zadnji projekt \sn{Svetlobni vodnjak} \cite{video} uporablja senzor Kinect za zajem globinske slike \cite{andersen2012kinect}.

Reference v \textsc{Bib}\TeX\ formatu lahko enostavno prekopiramo iz Google u\v cenjaka, tako da v nastavitvah Google u\v cenjaka najprej pri "`Upravitelju bibliografskih podatkov"' izberemo \textsc{Bib}\TeX.
Pri vseh zadetkih se bo potem pokazala mo\v znost \sn{Uvozi v \textsc{Bib}\TeX}.

Z uporabo opcije
\verb+\usepackage[hyphens]{url}+ v spisku literature ni te\v zav z dolgimi spletnimi naslovi~\cite{bimgrwm}!

Kako se citira diploma~\cite{diploma}?

\sn{Slovenski narekovaji}.

Javno predavanje~\cite{src1}.

Spletna stran~\cite{nscu}.

\bibliographystyle{plain}
\bibliography{reference}



\end{document}  