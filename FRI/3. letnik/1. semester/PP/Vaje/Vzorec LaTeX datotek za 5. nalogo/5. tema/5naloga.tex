\documentclass[11pt,a4paper]{article}

\usepackage[utf8x]{inputenc}   % omogoča uporabo slovenskih črk kodiranih v formatu UTF-8
\usepackage[slovene]{babel}    % naloži, med drugim, slovenske delilne vzorce

\usepackage[hyphens]{url}
\usepackage{hyperref}


\title{Naslov možne diplomske naloge}
\author{Ime Priimek\\
e-naslov\\
\ \\
predvideni MENTOR: (viš. pred./doc./prof.) dr. X Y \\
Fakulteta za računalništvo in informatiko Univerze v Ljubljani
\date{\today}         
}



\begin{document}
\maketitle

\section{Najbolj relevantna publikacija mojega predvidenega mentorja v zvezi z mojo predvideno diplomsko nalogo}

V nekaj stavkih opiši vsebino te publikacije in kako se navezuje na temo predvidene diplomske naloge.
Navedite ta članek v seznamu literature in se nanj sklicujte v tem odstavku.

Za citiranje vse navedene literature uporabi  Bib\TeX!



\section{Katere so tri najbolj citirane publikacije mojega predvidenega mentorja}

\subsection{V sistemu COBISS oziroma SICRIS}

V sistemu SICRIS se spremlja le publikacije v zadnjih petih letih!

Citati pa se štejejo v zadnjih desetih letih!

Šteje se le citate v revijah in iz revij, ki so vključene v Web of Science.


\subsection{V Google učenjaku}

V Google učenjaku privzeta časovna doba ni omejena.

Šteje se vse citate, ki jih Google najde na internetu.


\section{Kakšen je h-indeks mojega predvidenega mentorja}

Številka odraža razliko, katere publikacije spremlja COBISS in katere Google učenjak, ter še upoštevano časovno obdobje!


\subsection{V sistemu SICRIS}

\subsection{V Google učenjaku}


\section{Strani mentorja na Google učenjaku, na Research gate in v  Academiji}

Preveri, ali ima mentor svojo stran v Research gate, Academiji in na Google učenjaku in podaj spletni naslov teh strani, če obstajajo!


Na primer, publikacije Franca Soline lahko najdemo na naslednji akademskih portalih:
\begin{description}
\item[Google učenjak:] \url{http://scholar.google.si/citations?user=ShKuoiUAAAAJ}

\item[Research gate:] \url{https://www.researchgate.net/profile/Franc_Solina/}

\item[Academia:] \url{http://uni-lj.academia.edu/FrancSolina}

\end{description}

\bibliographystyle{plain}
\bibliography{literatura}

\end{document}  




