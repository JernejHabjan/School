\documentclass[11pt,a4paper]{article}

\usepackage[utf8x]{inputenc}   % omogoča uporabo slovenskih črk kodiranih v formatu UTF-8
\usepackage[slovene]{babel}    % naloži, med drugim, slovenske delilne vzorce



\title{Naslov možne diplomske naloge}
\author{Ime Priimek\\
e-naslov\\
\ \\
MENTOR: (viš. pred./doc./prof.) dr. X Y \\
Fakulteta za računalništvo in informatiko Univerze v Ljubljani
\date{\today}         
}


\begin{document}
\maketitle

\section{Motiv za diplomsko nalogo}

Zakaj me ta tema zanima? Komu bodo rezultati diplomske naloge koristili?


\section{Kako so se s to temo dosedaj ukvarjali učitelji na FRI?}

Povzemi glavne rezultate in navedi ustrezne članke v seznamu literature!


\section{Kaj je konkretni cilj diplomske naloge in kateri so glavni koraki do tega cilja?}

Naštej po točkah bistvene korake do uresničitve cilja!


\section{S kakšnimi orodji boš prišel do cilja}

Kakšno opremo in orodja bomo uporabljali pri izdelavi diplomske naloge?


\section{Kako bomo preizkusili rešitev ali ustreza zadanim ciljem?}

V tem dokumentu, ki naj bo dolg vsaj dve polni strani in največ tri strani, je potrebno citirati vsaj pet člankov pedagogov naše fakultetete, ki so bili objavljeni v znanstveni reviji.

Primer navajanja članka v reviji~\cite{ravnik2013audience}.


\section{Zaključek: zakaj je izbrani mentor primeren za predlagano temo?}

Članki? Nosilec predmeta? Vodja laboratorija?

\bibliographystyle{plain}
\bibliography{literatura}

\end{document}  




