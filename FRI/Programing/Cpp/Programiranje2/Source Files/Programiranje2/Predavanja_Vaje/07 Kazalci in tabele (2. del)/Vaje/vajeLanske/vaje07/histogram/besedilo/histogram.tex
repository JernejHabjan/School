\naslovI{Histogram besed}

\besedilo

Napišite program, ki za podano besedilo izpiše $n$ najbolj pogostih besed.
Besede morajo biti izpisane v padajočem vrstnem redu glede na število
pojavitev, zraven besede izpišite njeno frekvenco. Če se več besed
pojavi v besedilu enako pogosto, jih uredite po abecedi.

Nalogo bomo reševali po korakih. Imeli bomo
tri (pod)naloge. Za vse naloge imate ločene teste.

\subsubsection*{Naloga1}
Preberite besede (uporabite \texttt{\%s} pri \texttt{scanf}) in jih izpišite.
Pri tem upoštevajte:
\begin{itemize}
 \item Vsa ločila je potrebno izločiti (uporabite funkcijo ispunct). Ena beseda
ima lahko tudi več ločil.
 \item Velike in male črke se obravnavajo enako. Tako se
v frazi "Gori na gori gori" beseda "gori" pojavi 3x, čeprav je 1x zapisana z
\% veliko začetnico in 2x z malo.
 \item Lahko predpostavite, da bodo v besedilih le ASCII znaki.
\end{itemize}

\subsubsection*{Naloga2}
Isto kot naloga1, vendar se vsako besedo izpiše le enkrat. Če se beseda v
tekstu ponovi, je ne izpisujte.

\subsubsection*{Naloga3}
Rešite celotno nalogo. Namig: uporabite še eno tabelo za štetje besed, ki bo imela enako število elementov, kot jih ima tabela z besedami.

\vhod

V vhodnih podatkih bo najprej napisano zahtevano število najbolj pogostih besed,
sledi besedilo. Vhodni podatki se nahajajo v direktoriju \texttt{tests}.

\begin{primerVhod}
5
Levo, pa desno, pa spet levo in potem naravnost do semaforja, kjer zavijes spet
levo. Pa pazi na pesce!
\end{primerVhod}

Vhodne besede ne bodo daljše od $10^5$ znakov.
Besedilo je lahko poljubne dolžine.


\izhod

Izhodne datoteke za posamezne naloge se nahajajo v direktorijih \texttt{out1-3}.

\subsubsection*{Naloga1}

\begin{primerIzhod}
levo
pa
desno
pa
spet
levo
in
potem
naravnost
do
semaforja
kjer
zavijes
spet
levo
pa
pazi
na
pesce
\end{primerIzhod}

\subsubsection*{Naloga2}

\begin{primerIzhod}
levo
pa
desno
spet
in
potem
naravnost
do
semaforja
kjer
zavijes
pazi
na
pesce
\end{primerIzhod}

\subsubsection*{Naloga3}

Vsako najbolj pogosto besedo skupaj s frekvenco izpišite v svoji vrstici, ki se
mora zakljuciti z izhodnim znakom za novo vrstico
'\textbackslash n'.

\begin{primerIzhod}
levo 3
pa 3
spet 2
desno 1
do 1
\end{primerIzhod}



